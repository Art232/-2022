\documentclass{beamer}
\usetheme{Madrid}

\usepackage[T2A]{fontenc}
\usepackage[utf8]{inputenc}
\usepackage[russian]{babel}
\usepackage{mathptmx} 

\title{THE MATRIX RELOADED}
\subtitle{Решение и разбор}
\author{Артём Куртев, Дмитрий Попков}
\institute{БФУ имени И.Канта}
\date{\today}
\begin{document}
	\begin{frame}
	\titlepage
	\end{frame}
	\begin{frame}
		\frametitle{Содержание}
		\begin{itemize}
			\item Разбор задачи
			\begin{itemize}
				\item Формулировка
				\item Исходных данных
			\end{itemize}
			\item Решение задачи
			\item Ссылки
		\end{itemize}
	\end{frame}
	
	\begin{frame}
		\frametitle{Разбор задачи}
		\begin{block}{Формулировка задачи}
			It's happening exactly as before... Well, not exactly.
		\end{block}
		\begin{block}{Исходные данные}
			\begin{itemize}
				\item generator.txt
				\item output.txt
				
			\end{itemize}
		\end{block}
	\end{frame}
\begin{frame}
		\frametitle{Решение задачи}
			\text{\large{\textbf{\textit{Код решения:}}}} \newline	
			(1)\text{\kern 1pc $from\,Crypto.Hash\,import\, SHA256$}\newline
			(2)\text{\kern 1pc $from\,Crypto.Util.number\,import *$}\newline
			(3)\text{\kern 1pc $from\,Crypto.Cipher\,import\,AES:$}\newline
			(4)\text{\kern 1pc $key_length = 128$}\newline
			(5)\text{\kern 1pc $key = SHA256.new(data=b'59598059112411096439149288006319$}\newline
                         \text{\kern 1pc $4479432167104358696183689094613629455477050781014885725186$}\newline
                         \text{\kern 1pc $9110638484873235200204605081157845088692257708370810040562$}\newline
                         \text{\kern 1pc $721345').digest()[:key_length]$}\newline
			(6)\text{\kern 1pc $iv = bytes.fromhex('334b1ceb2ce0d1bef2af9937cf82aad6')$}\newline
		        (7)\text{\kern 1pc $cipher = AES.new(key, AES.MODE_CBC, iv)$}\newline
		        (8)\text{\kern 1pc $cipher_text = bytes.fromhex('543e29415bdb1f694a705b2532a5beb$}\newline
                            \text{\kern 1pc $7ebd7009591503ef3c4fbcebf9e62fe91307e5d98efcd49f9f3b198595$}\newline
                            \text{\kern 1pc $ 6cafc89')$}\newline
			(9)\text{\kern 1pc $plaintext = cipher.decrypt(cipher_text))$}\newline
                        (10)\text{\kern 1pc $print(plaintext)$}\newline\newline
			\text{\small\textit{Использованный ЯП --- Python, версии 3.7}}
	\end{frame}
\begin{frame}
		\frametitle{Объяснение решения}
		(1) Подключается библиотека.\newline
		(2) Подключается библиотека.\newline
		(3) Подключается библиотека.\newline
		(4) Задаётся длина ключа.\newline
		(5) Получаем из строки новый объект при помощи преобразования .\newline
		(6) Создаём объект byets из строки шеснадцатиричных чисел (соль).\newline
		(7)Получаем шифр из ключа и соли ,\newline
		(8) Создаём объект byets из строки шеснадцатиричных чисел (шифртекст).\newline
		(9) Декодируем.\newline
               (10) На вывод получаем флаг.\newline
	\end{frame}

\begin{frame}
\frametitle{Ссылки}
		\begin{itemize}
			\item {GitHub ссылки
			\begin{itemize}
			\item \href{https://github.com/Art232/-2022}{GitHub-репозиторий}
			\item \href{https://goo-gl.me/g9NWZ}{Код}
			\item \href{https://github.com/Nelifax/Summer-practice/blob/7bea7073e4a0e0341c1cacd58ebccdce28801e29/LaTeX_work.tex}{LaTeX-код}
			\end{itemize}}
		\end{itemize}
	\end{frame}

\end{document}